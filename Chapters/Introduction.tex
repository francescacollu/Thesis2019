\chapter*{Introduction}
\addcontentsline{toc}{chapter}{Introduction}

\label{Introduction}

Open quantum systems are physical systems coupled with an environment with which they exchange information. Studying the dynamics of open systems is not an easy problem, due to the complexity of the interaction between the parts involved. In order to overcome this difficulty, over the decades several analytical e numerical methods have been developed. In addition to these more "traditional" approaches, over the last decades have been investigated and advanced the field of quantum simulators: experimental platforms that reproduce Hamiltonian models analyzable with difficulty. Their advantage is the fact that the quantum simulators are quantum systems that can be insulated from the environment or with parameters easily tunable in order to replicate the model under consideration.

In order to examine in depth systems that may be simulated, numerical strategies has been used and developed. In the present thesis, we examine three of the most employed numerical methods in the field of strongly correlated quantum systems: the corner-space renormalization method, the quantum trajectories method and the matrix product operators method. We have used these approach to study a XYZ Heisenberg spin-1/2 chain coupled to a pair of dissipators acting only on the edge of the system. 
\\


This thesis is organized as follows. 

In the first chapter we run the basic concepts of the theory of open systems, introducing the notion of pure and mixed states and the density matrix formalism. In particular, we talk about dynamics of open systems, described by a master equation derived under the condition of having a Markovian evolution for the system. In the end of the chapter, technical concepts concerning quantum systems are introduced. 

In chapter 2, we go through the most recent development in the field of quantum simulator concluding with some example of experimental platform that simulate spin systems. 

Chapter 3 deals with numerical approaches in solving open many-body system problems, with a detailed report about the three numerical methods used in the present work. 

Chapter 4 and chapter 5 contain the original part of the work; after a description of the model under study, some observables are analyzed: magnetization profile, two-point correlation function and spin current. In the last section, an investigation on the presence of a phase transition is performed.