\chapter*{Introduction}
\addcontentsline{toc}{chapter}{Introduction}
\label{Introduction}

The theory of open quantum systems, i.e. physical systems coupled to an environment with which they exchange information, is a backbone of modern research in quantum mechanics. The reason of this lies in the fact that having a system completely isolated from the environment is an idealization, because every real system is \emph{open}, thus it does not obey a perfectly unitary quantum dynamics. Indeed, the state of the system is not well represented by a ket vector which evolves obeying the Schr\"{o}dinger equation, but it is rather described by a density operator, the dynamics of which is determined by the Liouville-Von Neumann master equation.

Studying the dynamics of an open system is not an easy problem, because of the complexity of the interactions between the system and environment. Since the scientific world has been able to perform experiments on nanoscopic scale, where the laws of classical mechanics no longer give a correct description of the underlying physics, the need to overcome this limitation and to understand open quantum systems has become mandatory.

Over the years analytical and numerical strategies have been developed in order to tackle this highly non trivial task. In addition to these more ``traditional'' approaches, over the last two decades the field of quantum simulation has been revealing as a new and intriguing alternative: this paves the way to reproduce Hamiltonian models by means of suitable experimental platforms that can be controlled with a high degree of accuracy. Several proposals have been put in practice, as superconducting circuits with QED cavities or platforms that make use of ultracold atoms in optical lattices. These simulators also allow to realize spin systems; this is one of the reasons why studying them is particularly important.

We have chosen a boundary-driven Heisenberg chain: a prototypical interacting quantum model in one dimension. In particular, we have investigated the properties of a XYZ Heisenberg spin-1/2 chain coupled to a pair of dissipators acting only on the edges of the system. In such a system two kinds of dynamics compete: the Hamiltonian one and the dissipative one. Due to the complexity of treating a combination of interacting and dissipative problems, following an analytic route is generally unfeasible. Therefore, we have realized a purely numerical analysis. We have studied the steady-state as the asympotic long-time solution of the master equation mentioned above; in this state, we have analyzed three observables under different regimes of dissipation and of Hamiltonian evolution: magnetization profile, two-point correlation function, spin current.

In order to develop this analysis, three different numerical methods have been employed.  The fundamental use of the \emph{matrix product operators} (MPO) method has allowed us to compute the physical quantities for all lengths of the chain examined in this thesis (8, 12, 16 sites). A preexisting code has been adjusted for the problem under consideration. The \emph{quantum trajectories} (QT) method has been useful to compare the results obtained from MPO for 8-sites chain and to explore the plausibility of some results for 10-sites chain. The \emph{corner-space renormalization} (CSR) method has been studied and implemented but it seems not to be useful for this kind of quantum systems. In particular, it seems to work more properly when all the spins of the chain are coupled to dissipators. For the last two numerical methods, a code has been written and implemented from scratch, during this thesis work.
\\


This thesis is organized as follows. 

In the first chapter we run the basic concepts of the theory of open systems, introducing the notion of pure and mixed states and the density matrix formalism. In particular, we talk about dynamics of open systems, described by a master equation derived under the condition of having a Markovian evolution for the system. In the end of the chapter, technical concepts concerning quantum systems are introduced. 

In chapter 2, we go through the most recent development in the field of quantum simulator concluding with some examples of experimental platform that simulate spin systems. 

Chapter 3 deals with numerical approaches in solving open many-body system problems, with a detailed report about the three numerical methods used in the present work. 

Chapter 4 and chapter 5 contain the original part of the work; after a description of the model under study, three observables are analyzed: magnetization profile, two-point correlation function and spin current. In the last section, an investigation on the presence of a phase transition is performed. In the last chapter, there is a report on the performances of the three methods used, with explanations about the choice of the parameters in order to obtain converging results.

Finally, in the conclusions we summarize our results and discuss possible perspectives of this work.

The appendix A contains the analysis of the CSR algorithm and its pseudocode.