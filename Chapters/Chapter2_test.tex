% % Chapter 1

% \chapter{Overview of Numerical Renormalization Group Methods} % Main chapter title

% \label{Chapter1} % For referencing the chapter elsewhere, use \ref{Chapter1} 

% %----------------------------------------------------------------------------------------

% % Define some commands to keep the formatting separated from the content 
% \newcommand{\keyword}[1]{\textbf{#1}}
% \newcommand{\tabhead}[1]{\textbf{#1}}
% \newcommand{\code}[1]{\texttt{#1}}
% \newcommand{\file}[1]{\texttt{\bfseries#1}}
% \newcommand{\option}[1]{\texttt{\itshape#1}}

% %----------------------------------------------------------------------------------------
% \section{The problem}

% \section{The Numerical Renormalization Group}

% \section{The Density Matrix Renormalization Group}

% \section{The Corner-Space Renormalization Method}
% As seen in the previous sections, simulating large quantum systems is a challenging task because their complexity grows exponentially with their size. An interesting method that reduces this complexity is the one developed by~\cite{PhysRevLett.115.080604}: the corner-space renormalization method. 

% The name of this method refers to the idea of selecting a corner of the Hilbert space for a lattice system, using eigenvectors of the steady-state density matrix of smaller lattices. 

% In particular, let us see the steps on which the algorithm is based: 
% \begin{enumerate}
% \item determine the steady-state density matrix for small lattices, for which an exact diagonalization of the Liouville super-operator is possible;
% \item merge two lattices and select the M most probable product states as a basis for the corner space;
% \item determine the steady-state solution of the density matrix in the corner-space;
% \item increase the dimension M of the corner-space until the convergence is achieved;
% \item in order to create a larger lattice, go back to step 1.
% \end{enumerate}

