\chapter{Numerical Methods for Open Quantum Systems}
\label{Chapter2}
As we have seen in the previous chapter, the study of an open quantum system requires the knowledge of its density matrix, solution of the Lindblad master equation~\cite{presk:quant_info}:
\begin{equation*}
    \dot{\rho} = \mathcal{L}[\rho],
\end{equation*}
where
\begin{equation}
\label{eqn:lindblad_eqn}
    \mathcal{L}[\rho] \equiv -i[H, \rho] -\gamma \sum_{a>0}\Bigl(\frac{1}{2}L_a^{\dagger}L_a\rho + \frac{1}{2}\rho L_a^{\dagger}L_a - L_a\rho L_a^{\dagger}\Bigl),
\end{equation}
setting $\hbar = 1$.

In such a system the number of variables scales as the square of the dimension of the Hilbert space; for example, in a spin-$\frac{1}{2}$ system made up by $n$ elements, the size of the Hilbert space $\mathcal{H}$ is $2^n$ while the \emph{liouvillian dimension} equals to $2^{2n} \times 2^{2n}$. Clearly, a direct integration of the equation~\ref{eqn:lindblad_eqn} can be made only for systems with a very limited number of elements. One only needs to note the fact that for a brute-force integration of the equation~\ref{eqn:lindblad_eqn} for a system consisting of $8$ sites, about $34$ GB of memory would be necessary\footnote{This calculation considers the fact that the liouvillian contains, in general, complex numbers, each of them valued with a single precision (32 bit).}.

%The numerical methods employed in open quantum system problems can be classified in those that have a wave-function approach (quantum trajectories method, mean field method) and those that decompose the system into blocks (corner-space renormalization method, matrix product density operators method). 
In this section, we will examine some of the numerical techniques used for studying open quantum systems with very different approaches.

Two of the numerical methods analyzed in this chapter are essentially based on the Wilson's \emph{renormalization group} approach~\cite{RevModPhys.47.773} combined with the fundamental contribution of the White's \emph{density matrix renormalization group} (DMRG) method~\cite{s_white:dmrg} and are the \emph{corner-space renormalization} and the \emph{matrix product density operators} methods. They both use the idea of a system decomposition into blocks; even if the basic idea is the same, the developed approaches are actually different.

The third numerical method analyzed in the section~\ref{chapt2_qtm} is the \emph{quantum trajectories} method, based on the evolution of a Monte Carlo wave function.

\section{The Corner-Space Renormalization (CSR) Method}
\label{chapter2_csr}
The numerical methods based on renormalization group à la Wilson present a calculation problem due to the increase of the dimensions of the Hilbert space, while the blocks are merged; the fundamental aim of the corner-space renormalization method~\cite{PhysRevLett.115.080604} is to treat this problem.

The name of this method refers to the idea of selecting a \emph{corner} of the Hilbert space for a lattice system, using eigenvectors of the steady-state density matrix of smaller lattices.

Let us start considering two blocks of a quantum system consisting in a certain number $N$ of sites. The CSR approach is a recursive process that begins with the calculation of the steady-state density matrix of a small block (the first \emph{block} will be a single site, in the second iteration it will be a small lattice constituted of the two blocks emerged in the previous iteration, and so on); this calculation can be done studying the spectrum of the eigenvalue of the Liouvillian and taking the zero value, considered that
\begin{equation}
    \frac{d\rho}{dt} = \mathcal{L} \rho
\end{equation}
can be seen as an eigenvalue equation.

We now consider two spatially adjacent sites: the site A and the site B. We now call $\rho^{(A)}$ the steady-state density matrix of the first site and $\rho^{(B)}$ of the latter. In order to spatially merge these two sites, let us write $\rho^{(A)}$ and $\rho^{(B)}$ in terms of an orthonormal basis; if there is no degeneracy among eigevalues of $\rho^{(A)}$ and $\rho^{(B)}$, we can use the orthonormal basis formed by their eigenvectors. Instead, if there is degeneration, a Gram-Schmidt orthonormalization process can be used to obtain an orthonormal basis starting from the ensemble of eigenvectors.

In this way, we will have:
\begin{equation}
    \rho^{(A)} = \sum_r p_r^{(A)}\ket{\phi_r^{(A)}}\bra{\phi_r^{(A)}},
\end{equation}
\begin{equation}
    \rho^{(B)} = \sum_{r'} p_{r'}^{(B)}\ket{\phi_{r'}^{(B)}}\bra{\phi_{r'}^{(B)}}.
\end{equation}
The steady-state density matrix of the \emph{merged block} constituted of the block A and the block B will be the following:
\begin{align}
\label{rhoAUB}
    \rho^{(AUB)} &= \rho^{(A)} \otimes \rho^{(B)} = \\
    \label{rhoAUB_explicit}
    &= \sum_{r} \sum_{r'} p_r^{(A)}p_{r'}^{(B)}\ket{\phi_r^{(A)}}\ket{\phi_{r'}^{(B)}} \otimes \bra{\phi_{r'}^{(B)}}\bra{\phi_{r}^{(A)}}
\end{align}
At this point, the CSR method suggests to delineate the \emph{corner-space}; for this purpose, we consider the $M$ most probable product states coming from~\ref{rhoAUB} of the form $\ket{\phi_{r}^{(A)}}\ket{\phi_{r'}^{(B)}}$, i.e. we rank the~\ref{rhoAUB_explicit} according to the joint probability $p_r^{(A)}p_{r'}^{(B)}$. The $M$ states form an orthonormal basis that generates a subspace
\begin{equation}
\label{basis_corner}
    \Bigl\{\ket{\phi_{r_1}^{(A)}}\ket{\phi_{r_1'}^{(B)}}, \ket{\phi_{r_2}^{(A)}}\ket{\phi_{r_2'}^{(B)}}, \dots, \ket{\phi_{r_M}^{(A)}}\ket{\phi_{r_M'}^{(B)}} \Bigl\},
\end{equation}
called \emph{corner-space}, with
\begin{equation*}
    p_r_1^{(A)}p_{r_1'}^{(B)} \geq p_r_2^{(A)}p_{r_2'}^{(B)} \geq \dots \geq p_r_M^{(A)}p_{r_M'}^{(B)}.
\end{equation*}
So, now it becomes necessary doing a change of basis to the new block $A \cup B$, that is:
\begin{equation}
    H' = \tilde{O} H_{A\cup B} \tilde{O}^\dagger,
\end{equation}
where $\tilde{O}$ is a pseudo-orthogonal\footnote{The matrix $\tilde{O}$ is called pseudo-orthogonal because it is not a square matrix.} $M\times N$ matrix, where $N$ is the dimension of $H_{A\cup B}$; $\tilde{O}$ is formed by the $M$ eigenvectors of $\rho_{A\cup B}$ living in the corner-space. This density matrix can now be used to calculate the expectation values of observable.

The more $M$ is large, the more the results are exact, because the basis~\ref{basis_corner} spans a larger part of the Hilbert space. Therefore, at this point the value of $M$ should be increased until the convergence is reached.

In the work of~\cite{PhysRevLett.115.080604}, as in the present thesis, has been proved that convergence can be achieved with a number of states $M$ much smaller than the dimension of the Hilbert space.

%The resulting dimension of the space in which, from now on - re-iterating the procedure - the system will be studied, is made smaller.

\bigskip
In short, the algorithm is structured in the following stages, as sketched in the figure~\ref{fig:csr_sketch}:
\begin{description}
    \item[calculation] of the steady-state density matrix of a single block of the system;
    \item[merger] of two predetermined blocks;
    \item[expression] of the density matrix of the merged block in an orthonormal basis;
    \item[selection] of the M most probable states as a basis for the so-called \emph{corner-space};
    \item[increase] of the dimension M of the corner-space until the convergence of the observable is achieved.
\end{description}

\begin{figure}
    \centering
    \includegraphics[scale=0.5]{Figures/csr_method_sketch.png}
    \caption{Sketch of the corner space renormalization method.}
    \label{fig:csr_sketch}
\end{figure}

A detailed description of the implementation of the algorithm can be seen in~\ref{AppendixA}.


\section{The Quantum Trajectories (QT) Method}
\label{chapt2_qtm}
A completely different approach is reached in the \emph{quantum trajectories method}, in which pure states are the subjects of the study, instead of density matrices. This means that, if the Hilbert space has dimension $N$, the number of involved parameters ($\sim N$) is much smaller than the one required in calculations with density matrices ($\sim N^2$).

The idea of this method can be summarised in the following way.

First of all, given the~\ref{eqn:lindblad_eqn}, let us note that the \textbf{first three terms} of this equation can be regarded as the evolution performed by an effective non-Hermitian Hamiltonian, that is~\cite{PhysRevA.69.062317}:
\[
H_{\rm{eff}} = H_s + \rm{i}K,
\]
with
\[
K = -\frac{1}{2}\sum_\mu L_{\mu}^{\dagger}L_{\mu}.
\]
Indeed, we see that:
\[
    -i[H_{eff}, \rho] = -i[H_s, \rho] - \frac{1}{2}\sum_\mu \{L_{\mu}^{\dagger}L_{\mu}, \rho\}.
\]

The \textbf{last term} of the~\ref{eqn:lindblad_eqn}, is the one responsible for the so-called \emph{quantum jumps}; for this reason, the representation under which we have written the~\ref{eqn:lindblad_eqn} is called \emph{quantum jump picture}~\cite{PhysRevA.69.062317}. 

At an initial time $t_0$, the density matrix of the system is in a pure state
\[
\rho(t_0) = \ket{\phi(t_0)}\bra{\phi(t_0)};
\]
after a time $dt$, it evolves to the following statistical mixture:
\begin{equation}
    \rho(t_0 + dt) = \Bigl(1-\sum_\mu dp_\mu \Bigl)\ket{\phi_0}\bra{\phi_0} + \sum_\mu dp_\mu \ket{\phi_\mu}\bra{\phi_\mu},
\end{equation}
where
\begin{equation}
    dp_\mu = \bra{\phi(t_0)}L_{\mu}^{\dagger}L_{\mu}\ket{\phi(t_0)}dt
\end{equation}
is the probability that a jump occurs; in this case, the system evolves in the state
\begin{equation}
\label{eqn:phi_mu}
    \ket{\phi_\mu} = \frac{L_\mu}{\abs{L_\mu\ket{\phi(t_0)}}}\ket{\phi(t_0)}.
\end{equation}
Otherwise, if no jump happens, the system evolves according to the effective Hamiltonian $H_{eff}$ in the following way:
\begin{equation}
    \ket{\phi_0} = \frac{(1-\textrm{i}H_{eff}dt)}{\sqrt{1-\sum_\mu dp_\mu}}\ket{\phi(t_0)}.
\end{equation}

In order to decide if the jump occurs or not, a Monte Carlo method will be integrated in this picture. Namely, an uniform distribution in the unit interval $[0,1]$ is taken under consideration; for every experiment, a pseudo-random number $\epsilon$ is chosen from this uniform distribution, i.e. a coin is tossed: depending on the result of the throw, the possible situations are the following:
\begin{itemize}
    \item if $\epsilon < \sum_\mu dp_\mu$, the system jumps to one of the states $\ket{\phi_\mu}$, defined in~\ref{eqn:phi_mu}. In particular:
    \begin{itemize}
        \item if $0 \leq \epsilon \leq dp_1$, the system jumps to $\ket{\phi_1}$;
        \item if $dp_1 < \epsilon \leq dp_2$, the system jumps to $\ket{\phi_2}$;
        \item and so on;
    \end{itemize}
    \item if $\epsilon > \sum_\mu dp_\mu$, the system evolves to the state $\ket{\phi_0}$.
\end{itemize}

This process has to be repeated as many times as $n = \frac{T}{dt}$, where $T$ is the whole elapsed time during the evolution. Let us note that $dt$ must be taken much smaller than the scales relevant for the evolution of the system.

Every \emph{experiment}, i.e. every throw of the coin, gives a different \emph{quantum trajectory}, which can be used to calculate the mean value of an observable $A$ at a certain time $t$, in this way:
\begin{equation}
    \langle A(t)\rangle = \bra{\phi_i(t)}A\ket{\phi_i(t)}.
\end{equation}
Since this results from a Monte Carlo process, we can consider the mean value over $N$ \emph{experiments}:
\begin{equation}
    \langle A(t)\rangle = \lim_{N\to\infty} \frac{1}{N}\sum_{i=1}^{N}\bra{\phi_i(t)}A\ket{\phi_i(t)}.
\end{equation}


\section{The Matrix Product Density Operators (MPDO) Method}
Finally, we analyze the \emph{matrix product density operators} (MPDO) method.
This method is rooted in one of the first and most efficient numerical methods: the \emph{density matrix renormalization group} (DMRG)~\cite{s_white:dmrg}. The reason why the DMRG is a powerful method is because the many-body states of some 1D problems can be described in terms of the so-called \emph{matrix product states} (MPS)~\cite{PhysRevLett.93.207204, PhysRevLett:from_dmrg_to_mps}:
\begin{equation}
    \ket{\psi_{MPS}} = \sum_{i_1, ..., i_N = 1}^{d} \Tr(A^{[1]i_N}...A^{[N]i_N}) \ket{i_1,...,i_N},
\end{equation}
where the $A$'s are matrices with dimensions D and $d$ is the dimension of the Hilbert space of every sub-system $i_k$. The total number of parameters turns out to be $N \times d \times D^2$, i.e. it is linear in $N$.

The matrix product \emph{density operators} (MPDO) extend the MPS from pure to mixed states, following the idea explained, for example, in~\cite{PhysRevLett.93.207204}. 
\begin{equation}
\label{eqn:mpdo}
    \rho = \sum_{i_1, \dots ,i_N = 1}^{d} \sum_{j_1, \dots ,j_N = 1}^{d} \Tr\Bigl(A^{[1]i_1,j_1} \dots A^{[N]i_N,j_N}\Bigl) \ket{i_1, \dots ,i_N}\bra{j_1, \dots ,j_N},
\end{equation}
where $A^{[k]i_k,j_k}$ are $D^2 \times D^2$ matrices that can be written in this way:
\begin{equation}
    A^{[k] i,j} = \sum_{a = 1}^{d_k} B_k^{i,a} \otimes (B_k^{j,a})^*.
\end{equation}

Another way of considering the expression~\ref{eqn:mpdo} is the graphical notation to represent a tensor $A^{[k] i,j}$ sketched in figure~\ref{fig:tensor_network}. Here, every tensor is represented by a square with 4 legs: there are two of them for the bonds with the neighbours, the other two for the physical indices (the "bras" and "kets"). 

\begin{figure}[H]
    \centering
    \includegraphics[scale=0.30]{Figures/tensor_network.png}
    \caption{Graphical notation of tensors in a 1D system.}
    \label{fig:tensor_network}
\end{figure}

As regards dynamics, the formal solution of~\ref{eqn:lindblad_eqn} can be written as
\begin{equation}
\label{eqn:lindblad_dinamics}
    \rho(t) = e^{\hat{\mathcal{L}}t}\rho(0);
\end{equation}
we recall that $\hat{\mathcal{L}}$ is a Liouvillian super-operator defined in equation~\ref{eqn:lindblad_eqn}. As shown in~\cite{Prosen_2009, PhysRevLett.93.207205}, $\hat{\mathcal{L}}$ can be decomposed into terms that involve two contiguous sites, as
\begin{equation}
    \hat{\mathcal{L}}[\rho] = \sum_l \hat{\mathcal{L}}_{l,l+1}[\rho].
\end{equation}
For every time step $\tau$, we use the Trotter-Suzuki formula:
\begin{equation*}
    e^{\hat{\mathcal{L}}\tau} = \prod_k \exp{(\alpha_k\hat{\mathcal{L}}_1\tau)}\dots\exp{(\beta_k\hat{\mathcal{L}}_N\tau)} + \mathcal{O}(\tau^p),
\end{equation*}
with $p$ depending on the required accuracy.


%\noindent\rule[0.5ex]{\linewidth}{1pt}
%\noindent\rule[0.5ex]{\linewidth}{1pt}

%One can express a MPDO in terms of a pure state MPS using the \emph{purification}~\cite{nielsen_chuang} technique. The essential idea is to consider an auxiliary system with a Hilbert space of dimension $d_k$ and, after choosing an orthonormal basis $\ket{i_k, a_k}$, to write the corresponding MPS state as
%\begin{equation}
%    \ket{\psi} = \sum_{i_1,\dots,i_N} \sum_{a_1, \dots, a_N} Tr\Bigl(\prod_{k=1}^N B_k^{i_k, a_k}\Bigl) \ket{i_1a_1, \dots, i_Na_N}.
%\end{equation}
%and eventually the MPDO $\rho$ is obtained tracing over the indices referred to the auxiliary system $a_k$:
%\begin{equation}
%    \rho = Tr_a(\ket{\psi}\bra{\psi});
%\end{equation}
%this density matrix can be used to compute the expectation values of observables.