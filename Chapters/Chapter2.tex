\chapter{Numerical Methods for Open Quantum Systems}
The study of open systems requires to solve the Liouvillian-von Neumann master equation; solving this equation means that 

\section{The Corner-Space Renormalization Method (CSRM)}
The numerical methods based on renormalization à la Wilson present a calculation problem due to the increase of the dimensions of the Hilbert space, while the blocks are merged; the fundamental aim of the corner-space renormalization method~\cite{PhysRevLett.115.080604} is to treat this problem.

The name of this method refers to the idea of selecting a \emph{corner} of the Hilbert space for a lattice system, using eigenvectors of the steady-state density matrix of smaller lattices.

In particular, the algorithm is structured in the following stages:
\begin{description}
    \item[calculation] of the steady-state density matrix of a single block of the system;
    \item[merger] of two predetermined blocks;
    \item[expression] of the density matrix of the merged block in an orthonormal basis;
    \item[selection] of the M most probable states as a basis for the so-called \emph{corner-space};
    \item[increase] of the dimension M of the corner-space until the convergence of the observable is achieved.
\end{description}

This process has to be repeated as many times as it is needed to reach the dimensions of the system.


\section{The Quantum Trajectories Method (QTM)}
A completely different approach is reached in the \emph{quantum trajectories method}, in which pure states are the subjects of the study, instead of density matrices. This means that a vector of length $N$ (where $N$ is the dimension of the Hilbert space) is stored, rather than a matrix of dimensions $NxN$.

The idea of this method can be summarised in the following way.

First of all, given the [Lindblad equation from the incipit], let us note that the \textbf{first two terms} of this equation can be regarded as the evolution performed by an effective non-Hermitian Hamiltonian, that is~\cite{PhysRevA.69.062317}:
\[
H_{eff} = H_s + \rm{i}K,
\]
with
\[
K = -\frac{\hbar}{2}\sum_\mu L_{\mu}^{\dagger}L_{\mu}.
\]
In fact, we see that:
\[
    -\frac{\rm{i}}{\hbar}[H_{eff}, \rho] = -\frac{\rm{i}}{\hbar}[H_s, \rho] - \frac{1}{2}\sum_\mu \{L_{\mu}^{\dagger}L_{\mu}, \rho\}.
\]

The \textbf{last term} of the [Lindblad equation from the incipit], is the one responsible for the so-called \emph{quantum jumps}; for this reason, the representation under which we have written the [Lindblad equation from the incipit] is called \emph{quantum jump picture}. 

At an initial time $t_0$, the density matrix of the system is in a pure state $\rho_0 = \Ket{\phi(t_0)}\Bra{\phi(t_0)}$ 



\section{The Matrix Product Operators Method (MPOM)}
