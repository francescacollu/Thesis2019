\chapter{Numerical Methods for Open Quantum Systems}
As seen in the previous chapters, the study of open systems requires to solve the Liouvillian-von Neumann master equation; solving this equation means that 

\section{The Corner-Space Renormalization Method}
As seen in the previous sections, simulating large quantum systems is a challenging task because their complexity grows exponentially with their size. An interesting method that reduces this complexity is the one developed by~\cite{PhysRevLett.115.080604}: the corner-space renormalization method. 

The name of this method refers to the idea of selecting a corner of the Hilbert space for a lattice system, using eigenvectors of the steady-state density matrix of smaller lattices. 

In particular, let us see the steps on which the algorithm is based: 
\begin{enumerate}
\item determine the steady-state density matrix for small lattices, for which an exact diagonalization of the Liouville super-operator is possible;
\item merge two lattices and select the M most probable product states as a basis for the corner space;
\item determine the steady-state solution of the density matrix in the corner-space;
\item increase the dimension M of the corner-space until the convergence is achieved;
\item in order to create a larger lattice, go back to step 1.
\end{enumerate}

\section{The Quantum Trajectories Method}
\section{The Matrix Product Operators Method}
