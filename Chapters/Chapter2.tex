\chapter{Numerical Methods for Open Quantum Systems}
We can write the master equation in the \emph{lindbladian form}~\cite{presk:quant_info}:
\begin{equation}
    \dot{\rho} = \mathcal{L}(\rho),
\end{equation}
where
\begin{equation}
\label{eqn:lindblad_eqn}
    \mathcal{L}(\rho) = -\emph{i}[H, \rho] + \sum_{a>0}     \Bigl(-\frac{1}{2}C^{\dagger}_a C_a \rho - \frac{1}{2}\rho C_a^{\dagger}C_a + C_a\rho C^{\dagger}_a \Bigl).
\end{equation}

To study an open quantum system consisting of $m$ subsystems, each of them characterized by an Hilbert space of dimension $N$, it is necessary to determine $N^{2m}$ variables, i.e. the square of the size of the Hilbert space.

\section{The Corner-Space Renormalization Method (CSRM)}
The numerical methods based on renormalization à la Wilson present a calculation problem due to the increase of the dimensions of the Hilbert space, while the blocks are merged; the fundamental aim of the corner-space renormalization method~\cite{PhysRevLett.115.080604} is to treat this problem. \\
The name of this method refers to the idea of selecting a \emph{corner} of the Hilbert space for a lattice system, using eigenvectors of the steady-state density matrix of smaller lattices.

In particular, the algorithm is structured in the following stages:
\begin{description}
    \item[calculation] of the steady-state density matrix of a single block of the system;
    \item[merger] of two predetermined blocks;
    \item[expression] of the density matrix of the merged block in an orthonormal basis;
    \item[selection] of the M most probable states as a basis for the so-called \emph{corner-space};
    \item[increase] of the dimension M of the corner-space until the convergence of the observable is achieved.
\end{description}

This process has to be repeated as many times as it is needed to reach the dimensions of the system.


\section{The Quantum Trajectories Method (QTM)}
A completely different approach is reached in the \emph{quantum trajectories method}, in which pure states are the subjects of the study, instead of density matrices. This means that a vector of length $N$ (where $N$ is the dimension of the Hilbert space) is stored, rather than a matrix of dimensions $NxN$.

The idea of this method can be summarised in the following way.

First of all, given the~\ref{eqn:lindblad_eqn}, let us note that the \textbf{first three terms} of this equation can be regarded as the evolution performed by an effective non-Hermitian Hamiltonian, that is~\cite{PhysRevA.69.062317}:
\[
H_{eff} = H_s + \rm{i}K,
\]
with
\[
K = -\frac{\hbar}{2}\sum_\mu L_{\mu}^{\dagger}L_{\mu}.
\]
In fact, we see that:
\[
    -\frac{\rm{i}}{\hbar}[H_{eff}, \rho] = -\frac{\rm{i}}{\hbar}[H_s, \rho] - \frac{1}{2}\sum_\mu \{L_{\mu}^{\dagger}L_{\mu}, \rho\}.
\]

The \textbf{last term} of the~\ref{eqn:lindblad_eqn}, is the one responsible for the so-called \emph{quantum jumps}; for this reason, the representation under which we have written the~\ref{eqn:lindblad_eqn} is called \emph{quantum jump picture}~\cite{PhysRevA.69.062317}. 

At an initial time $t_0$, the density matrix of the system is in a pure state
\[
\rho(t_0) = \ket{\phi(t_0)}\bra{\phi(t_0)};
\]
after a time $dt$, it evolves to the following statistical mixture:
\begin{equation}
    \rho(t_0 + dt) = \Bigl(1-\sum_\mu dp_\mu \Bigl)\ket{\phi_0}\bra{\phi_0} + \sum_\mu dp_\mu \ket{\phi_\mu}\bra{\phi_\mu},
\end{equation}
where
\begin{equation}
    dp_\mu = \bra{\phi(t_0)}L_{\mu}^{\dagger}L_{\mu}\ket{\phi(t_0)}dt
\end{equation}
is the probability that a jump occurs; in this case, the system evolves in the state
\begin{equation}
    \ket{\phi_\mu} = \frac{L_\mu}{\abs{L_\mu\ket{\phi(t_0)}}}\ket{\phi(t_0)}.
\end{equation}
Otherwise, if no jump happens, the system evolves according to the effective Hamiltonian $H_{eff}$ in the following way:
\begin{equation}
    \ket{\phi_0} = \frac{(1-\textrm{i}H_{eff}dt/\hbar)}{\sqrt{1-\sum_\mu dp_\mu}}\ket{\phi(t_0)}.
\end{equation}

In order to decide if the jump occurs or not, a Monte Carlo method will be integrated in this picture. Namely, an uniform distribution in the unit interval $[0,1]$ is taken under consideration; for every experiment, a pseudo-random number $\epsilon$ is chosen from this uniform distribution, i.e. a coin is tossed: depending on the result of the throw, the possible situations are the following:
\begin{itemize}
    \item if $\epsilon < \sum_\mu dp_\mu$, the system jumps to one of the states $\ket{\phi_\mu}$. In particular:
    \begin{itemize}
        \item if $0 \leq \epsilon \leq dp_1$, the system jumps to $\ket{\phi_1}$;
        \item if $dp_1 < \epsilon \leq dp_2$, the system jumps to $\ket{\phi_2}$;
        \item and so on;
    \end{itemize}
    \item if $\epsilon > \sum_\mu dp_\mu$, the system evolves to the state \ket{\phi_0}.
\end{itemize}

This process has to be repeated as many times as $n = \frac{T}{dt}$, where $T$ is the whole elapsed time during the evolution. Let us note that $dt$ must be taken much smaller than the evolution time of the system.

Every \emph{experiment}, i.e. every throw of the coin, gives a different \emph{quantum trajectory}, which can be used to calculate the mean value of an observable $A$ at a certain time $t$, in this way:
\begin{equation}
    \langle A(t)\rangle = \bra{\phi_i(t)}A\ket{\phi_i(t)}.
\end{equation}
Since this is a result of a Monte Carlo process, a good idea to obtain a reasonable outcome is to repeat the \emph{experiment} a number $N$ of times and to take the mean value of the results:
\begin{equation}
    \langle A(t)\rangle = \lim_{N\to\infty} \frac{1}{N}\sum_{i=1}^{N}\bra{\phi_i(t)}A\ket{\phi_i(t)}.
\end{equation}


\section{The Matrix Product Operators Method (MPOM)}
