%One of the main goals in scientific research is coming up to quantum simulation; this is due to the fact that in several physical fields there are mechanisms that cannot be simulated by classical computers: the essential aim is using a controlled quantum device to investigate another quantum system. Experiments have shown that superconducting circuits are able to manipulate and measure at the level of a few qubits and microwave photons, as a quantum simulator~\cite{Nat.Phys2012}. They can be treated as open systems because of the loss of photons, coupled to the feeding in additional photons through continuous external driving. Moreover, they are greatly flexible, because of their nanofabricated nature and almost every parameter involved is widely tunable. In~\cite{PhysRevX.7.011016}, it has been demonstrated that QED cavity lattices act as a controllable platform guiding understanding of non-equilibrium physics. 

\chapter{Open Many-Body Quantum Systems Coupled to External Baths}
\textcolor{red}{Not ultimate!}
One of the main goals in scientific research is coming up to quantum simulation; this is due to the fact that in several physical fields there are mechanisms that cannot be simulated by classical computers: the essential aim is using a controlled quantum device to investigate another quantum system.

\section{Basics of Dynamics}
As expressed in the previous chapter, an open quantum system can be described in terms of $\rho$, the reduced matrix given by averaging over the environment: 
\begin{equation}
    \rho_S = \Tr_E(\rho).
\end{equation}

\textcolor{red}{This section will conclude saying that the complexity of the systems become unmanageable, eventually.}

\section{Quantum Simulators: Controllable Many-Body Systems}
In order to overcome the problem mentioned in the previous section, over the years several analytical and numerical methods have been developed. 
% quantum simulators
% QED cavities
% optical cavities: photon blockade effect
% Jaynes-Cummings nonlinearities
% pag. 55 Carusotto-Ciuti
% fig.10 di Noh2016

\section{Spin Systems as Quantum Simulators}
% pag.5 Tomadin-Fazio