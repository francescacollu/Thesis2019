\chapter{Open Quantum Systems}
\label{Chapter1}

% Define some commands to keep the formatting separated from the content 
\newcommand{\keyword}[1]{\textbf{#1}}
\newcommand{\tabhead}[1]{\textbf{#1}}
\newcommand{\code}[1]{\texttt{#1}}
\newcommand{\file}[1]{\texttt{\bfseries#1}}
\newcommand{\option}[1]{\texttt{\itshape#1}}

%----------------------------------------------------------------------------------------
The world we live in is constituted by parts that communicate with each other and exchange information \textcolor{red}{[EXAMPLES]}. The difficulty of describing it, lies in the complexity of the interactions between those parts. To overcome these problems, simplifications and mathematical representation have been brought about in this field: in the following we will see some of them. 

\section{Closed and Open Quantum Systems}
Before we explore the field of open quantum systems, let us observe what are the differences between a closed and an open quantum system and why different approaches are required to treat these two distinct kinds of system. First of all, let us define a \emph{closed} system as a physical system which does not exchange any information with its surroundings. On the contrary, an \emph{open} system is a physical system that interacts with the environment in which it is. 

In this section, we will briefly look at the different approaches in the study of the dynamics in closed and open quantum systems. 

\subsection{Pure and mixed states}
First of all, it is useful consider another formalism in addition to that used normally for handling pure states. We now introduce the \emph{density matrix} formalism for managing the so-called mixed states. The essential difference between this two kind of states 

\subsection{Closed Quantum Systems}
Quantum mechanics establishes that the time evolution of a state vector $\ket{\psi(t)}$ is predicted by the Schrödinger equation:
\begin{equation}
\label{eqn:schrod_eq}
    i\frac{d}{dt}\ket{\psi(t)} = H(t)\ket{\psi(t)},
\end{equation}
where $H(t)$ is the Hamiltonian of the closed system and Planck's constant $\hbar$ has been set equal to 1.
The solution of this equation can be written in terms of the unitary time-evolution operator $U(t, t_0)$:
\begin{equation}
\label{eqn:schrod_u}
    \ket{\psi(t)} = U(t, t_0)\ket{\psi(t_0)},
\end{equation}
where $t_0 < t$. If we substitute the equation~\ref{eqn:schrod_u} into the~\ref{eqn:schrod_eq}, we obtain:
\begin{equation}
    i\frac{d}{dt}U(t, t_0) = H(t)U(t, t_0),
\end{equation}
subject to the initial condition
\begin{equation}
    U(t_0, t_0) = \mathds{1}.
\end{equation}

\subsection{Open Quantum Systems}


\section{Open Quantum Systems: the Liouvillian-von Neumann \\Master Equation}
\section{Many-Body Open Quantum Systems}