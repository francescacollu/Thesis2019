\chapter{Open Quantum Systems}
\label{Chapter1}

% Define some commands to keep the formatting separated from the content 
\newcommand{\keyword}[1]{\textbf{#1}}
\newcommand{\tabhead}[1]{\textbf{#1}}
\newcommand{\code}[1]{\texttt{#1}}
\newcommand{\file}[1]{\texttt{\bfseries#1}}
\newcommand{\option}[1]{\texttt{\itshape#1}}

%----------------------------------------------------------------------------------------
The world we live in is constituted by parts that communicate with each other and exchange information \textcolor{red}{[EXAMPLES]}. The difficulty of describing it, lies in the complexity of the interactions between those parts. To overcome these problems, simplifications and mathematical representation have been brought about in this field: in the following we will see some of them. 

Several resources have been helpful in the \dots

\section{Closed and Open Quantum Systems}
Before we explore the field of open quantum systems, let us observe what are the differences between a closed and an open quantum system and why different approaches are required to treat these two distinct kinds of system. First of all, let us define a \emph{closed} system as a physical system which does not exchange any information with its surroundings. On the contrary, an \emph{open} system is a physical system that interacts with the environment in which it is. 

In this section, we will briefly look at the different approaches in the study of the dynamics in closed and open quantum systems. 

\subsection{Pure and mixed states: the density matrix formalism}
First of all, it is useful consider another formalism in addition to that used normally for handling pure states, i.e. states that can be described by a single wave function. We now introduce the \emph{density matrix} formalism for managing the so-called mixed states, although - as we will see - it can be used also for pure states.

Quantum systems prepared in such a way that their state vector being obtained performing a \emph{maximal measurement}, in the sense that the maximum possible information has been acquired (all values of a complete set of commuting observables have been ascertained), are in a pure state. 

Quantum systems for which the maximum possible information is not available, are said to be in mixed state and are called \emph{statistical mixtures}.

Let us consider a system consisting of an ensemble of N sub-systems $\alpha = 1, 2, \dots , N$. We suppose that every sub-system is in a pure state $\ket{\psi_\alpha}$. Then, we choose a complete set of basis vectors $\ket{n}$, such that $\sum_n \ket{n}\bra{n} = \mathds{1}$. Let us expand the pure states in the basis $\{\ket{n}\}$:
\begin{equation*}
    \ket{\psi_\alpha} = \sum_n c_n^{\alpha}\ket{n},
\end{equation*}
where the coefficients $c_n^{(\alpha)}$ are such that
\begin{equation*}
    \sum_n |c_n^{(\alpha)}|^2 = 1.
\end{equation*}
Now let us consider an observable represented by an operator $A$; its expectation value in the pure state $\psi_\alpha$ is:
\begin{align}
    \braket{A}_\alpha &= \bra{\psi_\alpha}A\ket{\psi_\alpha} \\
                      \label{braket_A}
                      &= \sum_n \sum_{n'} c_{n'}^{(\alpha)*} c_n^{(\alpha)} \bra{n'}A\ket{n} \\
                      &= \sum_n \sum_{n'} \braket{n|\psi_\alpha}\braket{\psi_\alpha|n'} \bra{n'}A\ket{n}.
\end{align}
The average value of A over the ensemble is
\begin{equation}
\label{stat_averageA}
    \braket{A} = \sum_{\alpha = 1}^N w_\alpha \langle A \rangle_\alpha,
\end{equation}
where the coefficients $w_\alpha$ are the statistical weight of the pure states $\ket{\psi_\alpha}$, i.e. the probability of finding the system in this state. 
So the $w_\alpha$ are such that
\begin{equation}
\label{charact_weights}
    0 \leq w_\alpha \leq 1
\end{equation}
and 
\begin{equation*}
    \sum_{\alpha=1}^N w_\alpha = 1.
\end{equation*}
Using the result~\ref{braket_A} in the~\ref{stat_averageA}, we obtain:
\begin{align}
    \braket{A} &= \sum_{\alpha = 1}^N \sum_n \sum_{n'} w_\alpha c_{n'}^{(\alpha)*} c_n^{(\alpha)} \bra{n'}A\ket{n} \\
                \label{total_stat_average}
               &= \sum_{\alpha = 1}^N \sum_n \sum_{n'} \braket{n|\psi_\alpha}w_\alpha \braket{\psi_\alpha|n'} \bra{n'}A\ket{n}.
\end{align}
We now introduce the \emph{density operator}
\begin{equation}
    \rho = \sum_{\alpha = 1}^N \ket{\psi_\alpha} w_\alpha \bra{\psi_\alpha},
\end{equation}
whose representation in the basis $\{n\}$ gives us the density matrix:
\begin{equation}
    \rho_{nn'} = \bra{n}\rho\ket{n'} = \sum_{\alpha = 1}^N w_\alpha c_{n'}^{(\alpha)*} c_n^{(\alpha)}.
\end{equation}
So, returning to the~\ref{total_stat_average} in the definition of the density matrix, we observe that:
\begin{equation*}
    \braket{A} = \sum_n \sum_{n'} \bra{n}\rho\ket{n'}\bra{n'}A\ket{n} = \Tr({\rho A}).
\end{equation*}
It is worth stressing the fact that the knowledge of the density matrix enables us to obtain the ensemble average of A.

Anyway, this leads us to the first of the three fundamental characteristics of the density matrix, that is:
\begin{equation}
    \Tr(\rho) = 1,
\end{equation}
that is true if we make the assumption that the $\ket{\psi_\alpha}$ are normalized to unity. If they are not, then the ensemble average of A is given by
\begin{equation}
    \braket{A} = \frac{\Tr(\rho A)}{\Tr(\rho)}.
\end{equation}
The second characteristic of the density matrix is that it is Hermitian, namely
\begin{equation}
    \bra{n}\rho\ket{n'} = \bra{n'}\rho\ket{n}^*.
\end{equation}
The third characteristic arises from the observation of the diagonal elements of $\rho$:
\begin{equation}
    \rho_{nn} = \bra{n}\rho\ket{n} = \sum_{\alpha = 1}^N w_\alpha |c_{n}^{(\alpha)}|^2,
\end{equation}
which tells us, with the~\ref{charact_weights}, that
\begin{equation}
    \rho_{nn} \geq 0,
\end{equation}
i.e. $\rho$ is a positive semi-definite matrix.
It is worth stress the physical interpretation of the diagonal elements; $w_\alpha$ is the probability of finding the system in the pure state $\psi_\alpha$, while $|c_{n}^{(\alpha)}|^2$ is the probability of finding $\psi_\alpha$ in the state $\ket{n}$. So, $\rho_{nn}$ gives the probability of finding a member of the ensemble in the state $\ket{n}$.

\subsection{Closed Quantum Systems}
\label{closed_systems}
Quantum mechanics establishes that the time evolution of a state vector $\ket{\psi(t)}$ is predicted by the Schrödinger equation:
\begin{equation}
\label{eqn:schrod_eq}
    i\frac{d}{dt}\ket{\psi(t)} = H(t)\ket{\psi(t)},
\end{equation}
where $H(t)$ is the Hamiltonian of the closed system and Planck's constant $\hbar$ has been set equal to 1.
The solution of this equation can be written in terms of the unitary time-evolution operator $U(t, t_0)$:
\begin{equation}
\label{eqn:schrod_u}
    \ket{\psi(t)} = U(t, t_0)\ket{\psi(t_0)},
\end{equation}
where $t_0 < t$. If we substitute the equation~\ref{eqn:schrod_u} into the~\ref{eqn:schrod_eq}, we obtain:
\begin{equation}
    i\frac{d}{dt}U(t, t_0) = H(t)U(t, t_0),
\end{equation}
subject to the initial condition
\begin{equation}
    U(t_0, t_0) = \mathds{1}.
\end{equation}
We can write the dynamics of a closed system using the density matrix formalism seen in the previous section. In order to do this, let us assume that at a initial time $t_0$ the state of the system is characterized by the density matrix
\begin{equation*}
    \rho(t_0) = \sum_{\alpha = 1}^N w_\alpha \ket{\psi_\alpha(t_0)} \bra{\psi_\alpha(t_0)},
\end{equation*}
that at a time $t$ evolves in this way:
\begin{equation*}
    \rho(t) = \sum_{\alpha = 1}^N w_\alpha U(t,t_0)\ket{\psi_\alpha(t_0)} \bra{\psi_\alpha(t_0)}U(t,t_0)^\dagger,
\end{equation*}
that is
\begin{equation*}
     \rho(t) = U(t,t_0) \rho(t_0) U(t,t_0)^\dagger.
\end{equation*}
Differentiating with respect to time the last equation we have
\begin{equation}
\label{eqn:motion_closed_dm}
    \frac{d}{dt}\rho(t) = -i[H(t), \rho(t)]
\end{equation}
i.e. an equation of motion for the density matrix, often called the \emph{Liouville-von Neumann equation}.


\subsection{Open Quantum Systems}
Now we can go into a more specific definition of open quantum systems. It can be defined as a system $S$ coupled with another quantum system $B$, called the \emph{environment}. Usually, the total system $S+B$ is treated as a closed system, so it follows Hamiltonian dynamics. The interactions between the two sub-systems cause an evolution in the state of the open system $S$, which can no longer be represented by unitary, Hamiltonian dynamics. Following the definitions given in~\cite{pet_breuer:open_quantum}, an environment with an infinite number of degrees of freedom such that the frequencies of the modes form a continuum, is called \emph{reservoir}. Also, when the reservoir is in thermal equilibrium state, is called \emph{bath}.

Let us call $\mathcal{H}_S$ and $\mathcal{H}_B$ the Hilbert spaces of the system and the environment, respectively. The Hilbert space of the combined system is given by the tensor product $\mathcal{H}_{S+B} = \mathcal{H}_S \otimes \mathcal{H}_B$. So, the total Hamiltonian can be written in this way:
\begin{equation}
    H(t) = H_S \otimes \mathds{1}_B + \mathds{1}_S \otimes H_B + H_I(t),
\end{equation}
where $H_S$ and $H_B$ are the free Hamiltonians of the system and the environment, respectively, and $H_I(t)$ is the Hamiltonian describing the interactions between $S$ and $B$.

At this point, we must find a relation between the density matrix $\rho \in \mathcal{H}$ and $\rho_S \in \mathcal{H}_S$; it is given by the partial trace over the environment $B$\textcolor{red}{[PROOF?]}:
\begin{equation}
    \rho_S = \Tr_B(\rho).
\end{equation}
This can help us in the study of the dynamics of the so-called \emph{reduced} density matrix $\rho_S$. Indeed, we can write:
\begin{equation}
    \rho_S(t) = \Tr_B\{\rho(t)\};
\end{equation}
observing that we assumed the total system to be \emph{closed}, it evolves unitarily so we have:
\begin{equation}
    \rho_S(t) = \Tr_B\{U(t, t_0)\rho(t_0)U(t, t_0)^\dagger\},
\end{equation}
where $U(t, t_0)$ is the time-evolution operator of the combined system $S+B$. So we can use the equation of motion~\ref{eqn:motion_closed_dm} and trace over the degrees of freedom of the environment, in order to obtain the equation of motion for the $\rho_S(t)$:
\begin{equation}
    \frac{d}{dt}\rho_S(t) = -i \Tr_B[H(t), \rho(t)].
\end{equation}

\section{Approximate Dynamics of Open Quantum Systems}

\subsection{The Dynamical Map}
\subsection{The Lindblad Master Equation}
\section{Many-Body Open Quantum Systems}