\chapter*{Conclusions}
\addcontentsline{toc}{chapter}{Conclusions}

\label{Conclusions}

In the presented work, we have focused on the study of strong correlated quantum systems driven on the edge by quantum baths. %steady-state system

One could wonder why it would be interesting study such a system: the answer comes from the breakthroughs in  experimental methods, e.g. in the field of ultracold atoms and QED cavities, the platforms that make it possible to reproduce Hamiltonian models not analyzeable otherwise and consequently to understand complex quantum-physical phenomena. 

In the work developed in the previous pages, we have concentrated upon a model, in particular: an anisotropic XYZ Heisenberg spin-1/2 chain driven far from equilibrium with a pair of dissipators acting on the edge of the chain only.

In such a system two kind of dynamics compete: the Hamiltonian one and the dissipative one. We have studied the steady-state to which the system tends while these two dynamics act; in this state, we have analyzed three observables under different regimes of dissipation and of Hamiltonian evolution: magnetization profile, two-point correlation function, spin current.

With regard to magnetization, we have obtained a non-linear trend that becomes more and more clear as the size of the chain increases. In the 16-sites chain, the longest we have studied, the middle part is characterized by zero magnetization and only in the edges it tends to increase; that is plausible because the driving effects prevail in that part of the chain. Moreover, while the dimension of the chain increases the plateau with zero magnetization grows. 

The two point correlation function, studied between equidistant spin from the center of the chain, reveals an exponential profile. Here, as in the analysis of the previous observable, the behaviour becomes more evident as the chain dimensions get bigger, because of the size effect.

The third physical quantity that we have examined is the spin current. The profile is polynomial, with two peak values corresponding to the currents between the first two and the last two spin of the chain: they are the maximum values because of the driving effects of the dissipators. From the edge values of spin current, it smoothly decreases. The middle area in which it reach the lowest values gets much flatter as the chain length grows. An interesting thing about the peak values of the spin current is the fact that they reach a maximum value while $\gamma$ increases and then decrease until they  asymptotically tend to zero.

In the second part of this thesis, we have also studied the spin current varying the coupling constant $J_z$ in the Hamiltonian. We discovered a discontinuous behaviour around $J_z = 0.5$: for $J_z < 0.5$, %(this behaviour is instead replayed by the spin current of the chain characterized by $J_z \geq 0.5$), 
the spins in the bulk of the chain (i.e. all of them excluding the first and the last) seem to produce a constant current; for $J_z > 0.5$, the spin current has the polynomial trend already observed. Even for the magnetization profile, we observe that the edges of the chain are characterized by the same values, for any $J_z<1$. However, for this observable we do not see any signature of discontinuity in the bulk, in contrast to the spin current. Instead, the two-point correlation function reveals another discontinuity again around $J_z = 0.5$: for $J_z < 0.5$, it is zero, while for $J_z > 0.5$, it has an exponential profile.

In order to develop this analysis, three different numerical methods have been employed.  The fundamental use of the MPO method has allowed us to compute the physical quantities for all lengths of the chain examined in this thesis (8, 12, 16 sites). A preexisting code has been adjusted for the problem under consideration. The QT method has been useful to compare the results obtained from MPO for 8-sites chain and to explore the plausibility of some results for 10-sites chain. The CSR method has been studied and implemented but it seems not to be useful for this kind of quantum systems. In particular, it seems to work more properly when all the spins of the chain are coupled to dissipators. For the last two numerical methods, a code has been written and implemented from scratch, during this thesis work.

%In order to develop this analysis studying three numerical methods has been necessary. The fundamental use of MPO method, the code of which is not part of the present original work, has allowed us to compute the physical quantities for all lengths of the chain examined in this thesis (8, 12, 16 sites). The QT method has been useful to compare the results obtained from MPO for 8-sites chain and to explore the plausibility of some results for 10-sites chain (here not reported). The CSR method has been studied and implemented but it seems not to be useful for this kind of quantum systems. In particular, it seems to work more properly when all the spins of the chain are coupled to dissipators. For the last two numerical methods, a code has been written and implemented.

%perché è interessante studiare sistemi dissipativi, interesse generale
%cosa abbiamo trovato
%ci siamo concentrati sullo studio dei sist quant fort corr in cui metto dissipaz ai bordi e vado a vedere sullo stato stazionario che sistema si forma, che caratteristiche ha: questo è interessante perché ci sono piattaforme sperimentali che permettono di studiare...
%nella tesi abbiamo scoperto che prendendo un heisenberg xyz si generano profili etc etc
%1-2 pagine