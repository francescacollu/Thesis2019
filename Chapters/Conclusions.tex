\chapter*{Conclusions}
\addcontentsline{toc}{chapter}{Conclusions} \markboth{Conclusions}{}

\label{Conclusions}

In the presented work, we have focused on the study of one dimensional strongly correlated quantum system driven on the two edges by quantum baths. %steady-state system

One could wonder why it would be interesting to study such kind of systems: the answer comes from the breakthroughs in  experimental methods, e.g. in the field of ultracold atoms~\cite{ultracoldAtoms_condMatter} and QED cavities~\cite{Tomadin_Fazio, Nat.Phys2012}, the platforms that make it possible to reproduce Hamiltonian models not analyzable otherwise and consequently to understand complex quantum-physical phenomena. 

In this thesis work, we concentrated on a specific model: the anisotropic XYZ Heisenberg spin-1/2 chain driven far from equilibrium with a pair of dissipators acting on the edge of the chain only.

In such a system two kind of dynamical mechanisms compete: the Hamiltonian one and the dissipative one. We have studied the steady-state to which the system tends to equilibrate, while these two mechanisms act; in this state, we have explicitly focused on three observables under different regimes of dissipation and of Hamiltonian evolution: the magnetization profile, the two-point correlation function and the spin current.

With regard to magnetization, we have obtained a non-linear trend that becomes more and more clear as the size of the chain increases. In the 16-sites chain, the largest we were able to study, the middle part is characterized by zero magnetization and only in the edges it tends to increase; that is plausible because the driving effects prevail in that part of the chain. Moreover, while the dimension of the chain increases, the plateau with zero magnetization grows. 

The two point correlation function, studied between equidistant spins from the center of the chain, reveals an exponential profile. Here, as in the analysis of the previous observable, the main features become more evident as the chain dimensions get bigger, because of the finite-size effects.

The third physical quantity that we have examined is the spin current. The profile is polynomial, with two peak values corresponding to the currents between the first two and the last two spins of the chain: they are the maximum values because of the driving effects of the dissipators. From the edge values of spin current, it smoothly decreases. In the bulk of the chain, in which it reaches the lowest values, it gets flatter as the chain length grows. An interesting thing about the peak values of the spin current is the fact that they reach a maximum value while $\gamma$ increases and then decrease until they  asymptotically tend to zero.

In the second part of this thesis, we have also studied the spin current varying the coupling constant $J_z$ in the Hamiltonian. We discovered a discontinuous behaviour around the critical value $J_z = J_z^{(c)}$: for $J_z < J_z^{(c)}$, %(this behaviour is instead replayed by the spin current of the chain characterized by $J_z \geq 0.5$), 
the spins in the bulk of the chain (i.e. all of them excluding the first and the last) seem to produce an approximately constant current; for $J_z > J_z^{(c)}$, the spin current has the polynomial trend already observed. Even for the magnetization profile, we observe that the edges of the chain are characterized by the same values, for any $J_z<1$. However, for this observable we do not see any signature of discontinuity in the bulk, in contrast to the spin current. Instead, the two-point correlation function reveals another discontinuity again around $J_z = J_z^{(c)}$: for $J_z < J_z^{(c)}$, it is compatible to zero, while for $J_z > J_z^{(c)}$, it is finite and exhibits an exponentially decaying profile.

In order to develop this analysis, three different numerical methods have been employed.  The substantial use of the MPO method~\cite{MPO_method} has allowed us to compute the physical quantities for all lengths of the chain examined in this thesis (8, 12, 16 sites). A preexisting code has been adjusted for the problem under consideration. The QT method~\cite{PhysRevLett.68.580} has been useful to compare the results obtained from MPO for 8-sites chain and to explore the plausibility of some results for 10-sites chain. The CSR method~\cite{PhysRevLett.115.080604} has been studied and implemented but it seems not to be useful for this kind of quantum systems. In particular, it seems to work more properly when all the spins of the chain are coupled to dissipators. For the last two numerical methods, a code has been written and implemented from scratch, during this thesis work.

In conclusion, with the present work we have performed an analysis of a boundary-driven interacting quantum spin system, that
has allowed us to better understand such kind of systems, but it also raises some open questions which encourage future investigations.

First of all, a deeper analysis of the conjectured quantum phase transition over $J_z$ would be desirable, through further extensive numerical simulations on chains with bigger sizes, that require longer computational time.

With regard to numerical approaches, an improvement could be developed on the available methods: we have seen that the CSR method is not suitable for this kind of systems, so developing alternative numerical strategies that give better results is to be hoped.
%pensare invece del CSR pensare a strategie alternative che potrebbero dare risultati migliori. 

Some further investigations can be done also on the modelling of the reservoirs:  in this thesis we have assumed the hypothesis of Markovian evolution for the system. Working on a different formalism from the Lindblad master equation can lead to a more complex, yet realistic, modelling of the baths coupled to the system. 

Speaking of baths, one could inquire what happens if each of the reservoirs is coupled not only to a single spin but to more than one: the results can be considerably different or qualitatively the same than the ones obtained in the present thesis. One could introduce disorder (for example, one could consider a model characterized by non-homogeneous coupling constants $J_x$, $J_y$ and $J_z$) which is expected in real systems.

Another interesting aspect that could be investigated in further analyses is the effect of the dimensionality on the properties of the system; for example, in two dimensions the correlations may have a different trend from the exponential one of the chain, because of the possible establishment of a long-range order, which is missing in the one dimensional system studied in the present thesis. 

We are confident that the analysis performed in this thesis will be useful as a starting point not only for further theoretical investigations, but also for the comprehension of future experiments with quantum simulators of open systems.

%siamo fiduciosi che l'analisi eseguita in questa tesi potrà servire come punto di partenza non solo per ulteriori investigazioni teoriche, ma anche per la comprensione di esperimenti futuri con i simulatori quantistici di sistemi aperti.


%Nelle prospettive parlerei di:
%- analisi più dettagliata della possibile presenza della transizione di fase in J_z. Puoi dire che occorrono simulazioni estensive a taglie più grandi, che richiedono tempo.
%- robustezza dei risultati che hai trovato se uno perturba il sistema, ad esempio introducendo del disordine (che è fisiologico nelle realizzazioni sperimentali).
%- estensione dell'analisi in dimenisonalità più alte, ad esempio in 2 dimensioni, dove probabilmente gli andamenti esponenziali possono essere sostituiti da altre caratteristiche, dovute alla presenza di ordinamento a lungo raggio.
%- perfezionamento dei metodi numerici a disposizione, per migliorare l'analisi numerica.
%- modellizzazione più realistica dei bagni (ad esempio con formalismo diverso dalla master equation alla Lindblad, che tenga in considerazione possibili effetti non Markoviani sul sistema)
%- interazione con i bagni estesa a più siti vicini ai bordi (non semplicamente uno solo, ma due o più per bordo). Cambierà qualcosa o i risultati sono qualitativamente gli stessi?