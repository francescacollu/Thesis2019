\chapter{The Corner-Space Renormalization Algorithm} % Main appendix title
\label{AppendixA}

The code has been written in C++ language, useful in this work especially because of the possibility of implementing an object-oriented programming; a great help has been given by the linear algebra library \emph{Armadillo}~\cite{arma:book, arma:art}. 

\begin{algorithm}

\SetAlgoLined
\KwData{
\begin{description}
    \item[N] is the number of iteration necessary to cover the entire system;
    \item[M] is the dimension of the corner-space; 
\end{description}
}
\tcc{The code starts in the following lines.}
\For{i = 1 \to N}
{
determine the steady-state density matrix of two spatially-adjacent blocks, $\rho^{(A)}$ and $\rho^{(B)}$;\\
express $\rho^{(A)}$ and $\rho^{(B)}$ in terms of a orthonormal basis:
$\rho^{(A)} = \sum_r p_r^{(A)}\ket{\phi_r^{(A)}}\bra{\phi_r^{(A)}}$ and $\rho^{(B)} = \sum_r p_r^{(B)}\ket{\phi_r^{(B)}}\bra{\phi_r^{(B)}}$;

\eIf{there is degeneration among eigenvalues}{apply Gram-Schmidt orthonormalization process}{use the basis of $\rho^{(A)}$'s and $\rho^{(B)}$'s eigenvectors}

merge the two density matrices $\rho^{(A)} \otimes \rho^{(B)}$;\\
rank the terms of $\rho^{(AUB)}$ according to the joint probability $p_r^{(A)}p_{r'}^{(B)}$;\\
take the $M$ most probable product states;\\
\eIf{M>physical Hilbert space dimension}{the loop restarts}{cut the Hilbert space with a pseudo-orthogonal matrix with dimensions $N \times M$}
}

\caption{The corner-space renormalization algorithm written in pseudocode.}
\label{pseudocode}
\end{algorithm}

One should noticed the usefulness of object-oriented programming in the work of turning this algorithm in code. Indeed, the code written above is actually useful only if the physical system under consideration is invariant under translations, i.e. all sites are the same.

In most cases (as that ones considered in~\ref{Chapter3}) 