\chapter{The Corner-Space Renormalization Algorithm} % Main appendix title
\label{AppendixA}

The code has been written in C++, useful in this work especially because of the possibility of implementing an object-oriented programming and employs \emph{Armadillo}~\cite{arma:book, arma:art} as linear algebra library. 

\begin{algorithm}

\SetAlgoLined
\KwData{
\begin{description}
    \item[d] is the number of sites;
    \item[N] is the number of iteration necessary to cover the entire system, i.e. $N = log_2d$;
    \item[M] is the dimension of the corner-space; 
\end{description}
}
\bigskip
\For{i = 1 \to N}
{
\smallskip
determine the steady-state density matrices $\rho^{(A)}$ and $\rho^{(B)}$ of two spatially-adjacent blocks;\\
\smallskip
express $\rho^{(A)}$ and $\rho^{(B)}$ in terms of an orthonormal basis:
$\rho^{(A)} = \sum_r p_r^{(A)}\ket{\phi_r^{(A)}}\bra{\phi_r^{(A)}}$ and $\rho^{(B)} = \sum_r p_r^{(B)}\ket{\phi_r^{(B)}}\bra{\phi_r^{(B)}}$;
\smallskip

\eIf{there is degeneration among eigenvalues}{apply Gram-Schmidt orthonormalization process}{use the basis of $\rho^{(A)}$'s and $\rho^{(B)}$'s eigenvectors}
\smallskip

merge the two density matrices $\rho^{(A)} \otimes \rho^{(B)}$;\\
\smallskip
rank the terms of $\rho^{(AUB)}$ according to the joint probability $p_r^{(A)}p_{r'}^{(B)}$;\\
\smallskip
take the $M$ most probable product states;\\
\smallskip
\eIf{M>physical Hilbert space dimension}{the loop restarts}{cut the Hilbert space with a pseudo-orthogonal matrix with dimensions $N \times M$}
}

\caption{The corner-space renormalization algorithm written in pseudocode.}
\label{pseudocode}
\end{algorithm}

One should notice the usefulness of object-oriented programming in the work of turning this algorithm in code. Indeed, the code written above is actually useful only if the physical system under consideration is invariant under translations, i.e. the A and B blocks are the same. In most cases (as that ones considered in~\ref{Chapter3}) there are no symmetries; for dealing with these circumstances, the creation of classes is a great support.

In particular, two classes have been created:
\begin{itemize}
    \item a \textbf{SITE} class: an object of this class has the essential property of the presence (or the absence) of a dissipator, which is characterized by a certain kind of it, e.g. a ladder operator (raising or lowering);
    \item a \textbf{BLOCK} class: a \emph{block} is an object made up by a certain number of objects \emph{site} (you can say it is a nested class).
\end{itemize}

