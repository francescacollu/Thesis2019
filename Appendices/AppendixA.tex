\chapter{The Corner-Space Renormalization Algorithm} % Main appendix title
\label{AppendixA}

The purpose of this section is to analyze the code based on the corner-space renormalization algorithm~\cite{PhysRevLett.115.080604} explained in section~\ref{chapter3_csr}.

The code has been written in C++, language useful in this work especially because of the possibility of implementing an object-oriented programming, and employs \emph{Armadillo}~\cite{arma:book, arma:art} as linear algebra library. 
\medskip

\begin{algorithm}[H]
\label{algo_trans_inv}
\SetAlgoLined
\DontPrintSemicolon
\KwIn{
\begin{description}
    \item[n_s] is the number of sites;
    \item[N_{iter}] is the number of iteration necessary to construct the entire system, i.e. $N_{iter} = log_2n_s$ in the case of a spin-$\frac{1}{2}$ systems;
    \item[M] is the dimension of the corner-space; 
    \item[H] is the Hamiltonian of the system;
\end{description}
}
\bigskip
\For{i = 1 \to $N_{iter}$}{
determine the steady-state density matrices of two spatially-adjacent blocks\;
express the density matrices in terms of an orthonormal basis\;

\eIf{there is degeneration among eigenvalues}{apply Gram-Schmidt orthonormalization process}{use the basis formed by the eigenvectors of the density matrices}

merge the two density matrices\;
express the merged density matrix in terms of an orthonormal basis\;
get the pseudo-orthogonal matrix formed by the $M$ most probable product states\;
\eIf{M > Hilbert space dimension}{
the loop restarts\;
}{do the change of basis\;}
the (new) merged block replaces the old ones\;
}

\caption{The CSR algorithm for 1D translation invariant systems.}
\label{pseudocode}
\end{algorithm}

One should notice the usefulness of object-oriented programming in the work of turning this algorithm in code. Indeed, the pseudocode written above is actually useful only if the physical system under consideration is invariant under translations, i.e. the merging blocks are the same. In most cases (as that ones considered in chapter~\ref{Chapter3}) there are no symmetries; for dealing with these circumstances, the creation of classes is a great support.

In particular, two classes have been created:
\begin{itemize}
    \item a \textbf{SITE} class: an object of this class has the essential property of the presence (or the absence) of a dissipator, which is characterized by a certain kind of it, e.g. a ladder operator (raising or lowering);
    \item a \textbf{BLOCK} class: a \emph{block} is an object made up by a certain number of objects \emph{site} (you can say it is a nested class). Every time a merge between two blocks is done, a new block emerges. It is characterized by the spatial position: in a merge it is important to know if it is a left or a right block.
\end{itemize}
In the algorithm~\ref{alg:csr_algo_gen}, a more general version of the code is proposed; the presence of the objects mentioned above allows to treat systems that do not show any symmetry.
\medskip
\begin{algorithm}[H]
\DontPrintSemicolon
\SetAlgoLined
\KwIn{
\begin{description}
    \item[n_s] is the number of sites;
    \item[H] is the Hamiltonian of the system;
    \item[M] is the dimension of the corner-space; 
    \item[C_i] are the dissipators, with the specification of type and site index (i).
\end{description}
}
\smallskip
\KwData{
\begin{description}
    \item[n_B] is the number of blocks.
\end{description}
}
\bigskip
\For{$i=0 \to n_s$}{
push site with index $i$ in the block\;
\While{$n_B \neq 1$}{
\For{$i=0 \to n_B$}{
\eIf{$i\%4 = 0$}{
merge the blocks B[i] and B[i+1] putting B[i] spatially on the left\;
}{merge the blocks B[i] and B[i+1] putting B[i] spatially on the right\;}
express the density matrix of the merged block in terms of an orthonormal basis\;
\eIf{there is degeneration among eigenvalues}{apply Gram-Schmidt orthonormalization process}{use the basis formed by the eigenvectors of the density matrix}
get the pseudo-orthogonal matrix formed by the $M$ most probable product states\;
change of basis through this pseudo-orthogonal matrix\;
get the density matrix in the corner-space\;
calculate the expectation values of observables\;
the (new) merged blocks replace the old ones\;}
}
}

\caption{The CSR algorithm for 1D systems.}
\label{alg:csr_algo_gen}
\end{algorithm}

