\chapter{The Corner-Space Renormalization Algorithm} % Main appendix title
\label{AppendixA}

The purpose of this section is to analyze the code based on the corner-space renormalization algorithm~\cite{PhysRevLett.115.080604} explained in~\ref{chapter2_csr} and used to get the results shown in~\ref{Chapter3}.

The code has been written in C++, useful in this work especially because of the possibility of implementing an object-oriented programming, and employs \emph{Armadillo}~\cite{arma:book, arma:art} as linear algebra library. 

\begin{algorithm}[H]
\label{algo_trans_inv}
\SetAlgoLined
\DontPrintSemicolon
\KwIn{
\begin{description}
    \item[n_s] is the number of sites;
    \item[N_{iter}] is the number of iteration necessary to cover the entire system, i.e. $N_{iter} = log_2n_s$;
    \item[M] is the dimension of the corner-space; 
    \item[H] is the Hamiltonian of the system;
\end{description}
}
\bigskip
\For{i = 1 \to $N_{iter}$}{
determine the steady-state density matrices $\rho^{(A)}$ and $\rho^{(B)}$ of two spatially-adjacent blocks\;
express $\rho^{(A)}$ and $\rho^{(B)}$ in terms of an orthonormal basis\;

\eIf{there is degeneration among eigenvalues}{apply Gram-Schmidt orthonormalization process}{use the basis of $\rho^{(A)}$'s and $\rho^{(B)}$'s eigenvectors}

merge the two density matrices $\rho^{(A)} \otimes \rho^{(B)}$\;
rank the terms of $\rho^{(AUB)}$ according to the joint probability $p_r^{(A)}p_{r'}^{(B)}$\;
take the $M$ most probable product states\;
do the change of basis\;
\eIf{M > Hilbert space dimension}{
the loop restarts\;
}{do the change of basis\;}
}

\caption{The CSR algorithm for 1D translation invariant systems.}
\label{pseudocode}
\end{algorithm}

One should notice the usefulness of object-oriented programming in the work of turning this algorithm in code. Indeed, the pseudocode written above is actually useful only if the physical system under consideration is invariant under translations, i.e. the A and B blocks are the same. In most cases (as that ones considered in~\ref{Chapter3}) there are no symmetries; for dealing with these circumstances, the creation of classes is a great support.

In particular, two classes have been created:
\begin{itemize}
    \item a \textbf{SITE} class: an object of this class has the essential property of the presence (or the absence) of a dissipator, which is characterized by a certain kind of it, e.g. a ladder operator (raising or lowering);
    \item a \textbf{BLOCK} class: a \emph{block} is an object made up by a certain number of objects \emph{site} (you can say it is a nested class). Every time a merge between two blocks is done, a new block emerges. It is characterized by the spatial position: in a merge it is important to know if it is a left or a right block.
\end{itemize}
In the~\ref{csr_algo_gen}, a more general version of the code is proposed; the presence of the objects mentioned above allows to treat systems that do not show any symmetry.

\begin{algorithm}[H]
\label{csr_algo_gen}
\DontPrintSemicolon
\SetAlgoLined
\KwIn{
\begin{description}
    \item[n_s] is the number of sites;
    \item[H] is the Hamiltonian of the system;
    \item[M] is the dimension of the corner-space; 
    \item[C_k] are the dissipators, with the specification of type and site index (k).
\end{description}
}
\smallskip
\KwData{
\begin{description}
    \item[n_B] is the number of blocks.
\end{description}
}
\bigskip
\For{$i=0 \to n_s$}{
push site with index $i$ in the block\;
\While{$n_B \neq 1$}{
\For{$i=0 \to n_B$}{
\eIf{$i\%4 = 0$}{
merge the blocks B[i] and B[i+1] putting B[i] spatially on the left\;
}{merge the blocks B[i] and B[i+1] putting B[i] spatially on the right\;}
change of basis\;
get the density matrix\;
calculate the expectation values of observables\;
the (new) merged blocks replace the old ones\;}
}
}

\caption{The CSR algorithm for 1D systems.}
\end{algorithm}

